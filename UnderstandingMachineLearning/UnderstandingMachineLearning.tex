\documentclass[UTF8]{ctexart}
% use package here
\usepackage{mathrsfs}
\usepackage{extarrows}
\usepackage{amsmath}
\usepackage{amssymb}
\DeclareMathOperator*{\argmax}{arg\,max}
\usepackage{enumerate}
\usepackage{graphicx}





% information of front page 
\title{Understanding Machine Learning}
\author{Yuyang Zhang}
\date{\today}

% body
\begin{document}
\maketitle
\newpage

\tableofcontents
\newpage



% ============================================================
%
%			第一章 Why can machine learn
%
% ============================================================

\section{why can machine learn}
\subsection{基本符号}
\subsubsection{输入}
Domain Set:
一个任意集合$\mathcal{X}$,
也作领域集。
可以理解为所有样本的集合,
其中每个样本通常以一个能够表征其特征的向量表示。

Label Set:
标签集$\mathcal{Y}$。
样本所属于的类别,
通常二分类问题,
标签集为$\{0,1\}$或者是$\{-1,+1\}$。

Training Data:
训练数据$\mathcal{S}=\{(x_1,y_1),...,(x_m,y_m)\}$,
也叫训练集,样本集。

\subsubsection{输出}
Predicting Rule:
预测规则,$h:\mathcal{X}\rightarrow\mathcal{Y}$。
该规则是一个由样本集合到标签集的映射,
可以理解为预测器(predictor),
\textbf	{假设(hypothesis)},
分类器(classifier),等等。

\subsubsection{数据生成模型}
我们假定样本$\mathcal{S}$是由概率分布$\mathcal{D}$生成,
并且根据一个标记函数$f:\mathcal{X}\rightarrow\mathcal{Y}$,
来标记样本的类别,
对于任意$i=1,...,m$都有
$y_i=f(x_i)$,
然而我们并不知道概率分布$\mathcal{D}$与标记函数$f$,
我们的目标就是找一个合适的假设$h$,
令其与标记函数$f$可以对样本做出相同的标记。

\subsubsection{衡量标准}
我们定义分类误差为:
未能成功预测随机数据点正确标签的概率,
即对于随机的一个$x\in\mathcal{X}$,
$h(x)\neq f(x)$的概率。


定义$\mathcal{A}\subseteq\mathcal{X}$为一个领域子集,
$\mathcal{A}$中的任意实例$x\in\mathcal{A}$的出现概率由
$\mathcal{D}(\mathcal{A})$所决定。
通常,
我们称$\mathcal{A}$为一个事件,
$\mathcal{A}=\{x\in\mathcal{X}:\pi(x)=1\}$,
其中$\pi:\mathcal{X}\rightarrow\{0,1\}$,
表示样本是否被观测到。
我们也将$\mathcal{D}(\mathcal{A})$写作
$\mathbb{P}_{x\sim\mathcal{D}} [\pi(x)]$。
此时我们可以定义假设$h$的错误率为:
$$
\mathcal{L}_{\mathcal{D},f}(h)\xlongequal{def}
\mathbb{P}_{x\sim\mathcal{D}} [h(x)\neq f(x)]\xlongequal{def}
\mathcal{D}(\{x|h(x)\neq f(x)\})
$$
其中误差的测量是基于概率分布$\mathcal{D}$和标记函数$f$的,
$\mathcal{L}_{\mathcal{D},f}(h)$也称为泛化误差,
\textbf{损失}
或者$h$的真实误差。

\subsection{经验风险最小化}
机器学习的过程都是基于训练集$\mathcal{S}$的,
训练集$\mathcal{S}$由未知分布$\mathcal{D}$从
领域集$\mathcal{X}$中采样得出,并由标记函数$f$标记,
机器学习的输出是一个基于训练集$\mathcal{S}$的假设,
$h_{\mathcal{S}}:\mathcal{X}\rightarrow\mathcal{Y}$。

由于我们并不知道分布$\mathcal{D}$与标记函数$f$,
所以我们只能根据训练集$\mathcal{S}$来判断
我们所选择假设的表现,定义训练误差为:
$$
\mathcal{L}_{\mathcal{S}}\xlongequal{def}
\frac{|\{x_i|h(x_i)\neq y_i,i=1,..,m\}|}{m}
$$
训练误差也称作经验误差和经验风险。

因为训练集$\mathcal{S}$是领域$\mathcal{X}$的一个子集,
所以训练样本是真实世界的一个缩影,
正如概率统计的一个核心思想,
通过样本反映总体。
所以我们认为利用样本集寻找一个较好的假设是可行的,
即最小化训练误差$\mathcal{L}_{\mathcal{S}}(h)$,
这称之为经验风险最小化,
ERM(Experience Risk Minimize)。

\subsubsection{过拟合}
一个假设在训练集上效果优异,
但在真实世界中表现却糟糕,
这种现象称之为过拟合。
当我们过度追求经验风险最小化原则时,
我们就有可能面临过拟合的风险。

\subsection{归纳偏好}
虽然经验风险最小化会有过拟合的风险,
相比于抛弃这个原则,我们更愿意去修正这个原则,
考虑我们对于假设的归纳偏好。
我们通常的解决方案是根据我们定好的归纳偏好,
在有限的假设空间中去搜索所要用的假设,
这些假设的集合成为假设类,记为$\mathcal{H}$,
则我们的学习过程可以记为:
$$
{ERM}_{\mathcal{H}}(\mathcal{S})\in 
\argmax_{h\in\mathcal{H}}
\mathcal{L}_{\mathcal{S}}(h)
$$
我们常见的正则化,就是一种归纳偏好,
$L1$正则化表明我们的归纳偏好是更喜欢参数稀疏的假设。

\subsection{为什么可以学习到东西}
本节旨在说明:
当拥有足够多样本时,
在有限假设空间$\mathcal{H}$中,
经验风险最小化${ERM}_{\mathcal{H}}$原则不会出现过拟合,
即我们通过训练样本,
可以找到一个足够好的假设$h_s$,
在真实世界中表现也足够好。


\textbf{定义1.1 \textit{可实现性假设}
存在$h^*\in\mathcal{H}$,
使得$\mathcal{L}_{\mathcal{D},f}(h^*)=0$。}:


该假设意味着,对于随机样本集$\mathcal{S}$,
由概率分布$\mathcal{D}$采样,
由标记函数$f$标记,
以概率$1$使得$L_{\mathcal{S}}(h^*)=0$,
其中样本集$\mathcal{S}$中的样本是独立同分布的。
我们定义$h_{\mathcal{S}}$为对$\mathcal{S}$
利用${ERM}_{\mathcal{H}}$得到的结果:
$$
h_{\mathcal{S}}\in 
\argmax_{h\in\mathcal{H}}
\mathcal{L}_{\mathcal{S}}(h)
$$

因为样本集$\mathcal{S}$仍是
领域集$\mathcal{X}$的子集,
是根据分布随机得到的实例集合,
会有一定概率使得采样得到的样本不具有代表性,
并不能反映真实的总体情况,
所以根据样本集$\mathcal{S}$得到的假设$h_{\mathcal{S}}$
并不一定准确,
在真实世界中的表现有可能很差。
因此,我们选择一定程度的容忍,
容忍会有一定几率采样到不具有代表性的样本,
一般来说,我们定义采样得到不具有代表性样本的概率之多为$\delta$,
则$1-\delta$为置信参数。
同时对于假设的预测效果,我们能容忍的损失上限为$\epsilon$,
称为精度参数,如果
$\mathcal{L}_{\mathcal{D},f}(h_{\mathcal{S}})>\epsilon$,
那么这是一个差的假设,
反之,则是一个好的假设。

对于我们的样本集$\mathcal{S}$,最好的假设$h_{\mathcal{S}}$仍有
$\mathcal{L}_{\mathcal{D},f}(h_{\mathcal{S}})>\epsilon$,
则这组样本采样失败,
因为我们的${ERM}_{\mathcal{H}}$无法从$\mathcal{S}$中
学到有用的东西。
当样本集中所有样本$(x_1,...,x_m)$,
都不具有代表性,我们记为:
$$
\{x_i|x_i\in\mathcal{S},i=1,...m,
\mathcal{L}_{\mathcal{D},f}(h_{\mathcal{S}})>\epsilon
\}
$$
则采样失败的概率上界为:
$$
\mathcal{D}^m
(
\{x_i|x_i\in\mathcal{S},i=1,...m,
\mathcal{L}_{\mathcal{D},f}(h_{\mathcal{S}})>\epsilon
\}
)
$$
因为每个样本都是不具有代表性的样本,
所以是采样失败的概率的上界。
设$\mathcal{H}_B$为差的假设的集合:
$$
\mathcal{H}_B=
\{
h|\mathcal{L}_{\mathcal{D},f}(h)>\epsilon,
h\in\mathcal{H}
\}
$$
同时设:
$$
M=\{x|x\in\mathcal{S},
\exists h \in \mathcal{H}_B,
\mathcal{L}_{\mathcal{S}}(h)=0
\}
$$
为误导集,误导集使差的假设在训练样本$\mathcal{S}$
上表现良好,但在真实世界中表现较差。
因为有可实现假设
$$
\exists h^*,\mathcal{L}_{\mathcal{D},f}(h^*)=0
$$
且有
$$
h_{\mathcal{S}}\in 
\argmax_{h\in\mathcal{H}}
\mathcal{L}_{\mathcal{S}}(h)
$$
产生$\mathcal{L}_{\mathcal{D},f}(h_{\mathcal{S}})>\epsilon$
是因为样本集不好,
样本集采样失败,
所以,
当且仅当$\mathcal{S}\subseteq M$时,
才会出现$\mathcal{L}_{\mathcal{D},f}(h_{\mathcal{S}})>\epsilon$,
我们将其表示为:
$$
\{x_i|x_i\in\mathcal{S},i=1,...m,
\mathcal{L}_{\mathcal{D},f}(h_{\mathcal{S}})>\epsilon
\}\subseteq M
$$
所以我们有
$$
\mathcal{D}^m
(\{x_i|x_i\in\mathcal{S},i=1,...m,
\mathcal{L}_{\mathcal{D},f}(h_{\mathcal{S}})>\epsilon
\})
\leq \mathcal{D}^m(M)
$$
而$M$又可以写作:
$$
M=\bigcup_{h\in\mathcal{H}_B}
\{
x|x\in \mathcal{S},
\mathcal{L}_{\mathcal{S}}(h)=0
\}
$$
则:
$$
\mathcal{D}^m
(\{x_i|x_i\in\mathcal{S},i=1,...m,
\mathcal{L}_{\mathcal{D},f}(h_{\mathcal{S}})>\epsilon
\})
\leq \mathcal{D}^m(\bigcup_{h\in\mathcal{H}_B}
\{
x|x\in \mathcal{S},
\mathcal{L}_{\mathcal{S}}(h)=0
\})
$$
由$P(A\cup B)\leq P(A)+P(B)$得:
$$
\mathcal{D}^m(\bigcup_{h\in\mathcal{H}_B}
\{
x|x\in \mathcal{S},
\mathcal{L}_{\mathcal{S}}(h)=0
\})
\leq
\sum_{h\in\mathcal{H}_B}
\mathcal{D}^m(\{
x|x\in \mathcal{S},
\mathcal{L}_{\mathcal{S}}(h)=0
\})
$$
其中我们将$\mathcal{S}$拆开,
$$
\mathcal{D}^m(\{
x|x\in \mathcal{S},
\mathcal{L}_{\mathcal{S}}(h)=0
\}
=\mathcal{D}^m(\{
x_i|h(x_i)=f(x_i),
x_i\in \mathcal{S},
i=1,...m
\})
$$
$$
\mathcal{D}^m(\{
x|x\in \mathcal{S},
\mathcal{L}_{\mathcal{S}}(h)=0
\}
=\prod^m_{i=1}\mathcal{D}(\{
x_i|h(x_i)=f(x_i),
x_i\in\mathcal{S}
\})
$$
根据$1-\epsilon\leq e^{-\epsilon}$,
对于等号右边的连乘的每一项都有:
$$
\mathcal{D}(\{
x_i|h(x_i)=y_i,
x_i\in\mathcal{S}
\})
=1-\mathcal{L}_{\mathcal{D},f}(h)
\leq 1-\epsilon
$$
对于所有的$h\in\mathcal{H}_B$:
$$
\mathcal{D}^m(\{
x|x\in \mathcal{S},
\mathcal{L}_{\mathcal{S}}(h)=0
\}
\leq (1-\epsilon)^m
\leq e^{-\epsilon m}
$$
可得:
$$
\mathcal{D}^m
(\{x_i|x_i\in\mathcal{S},i=1,...m,
\mathcal{L}_{\mathcal{D},f}(h_{\mathcal{S}})>\epsilon
\})
\leq |\mathcal{H}_B|e^{-\epsilon m}
\leq |\mathcal{H}|e^{-\epsilon m}
$$
所以,若我们对采样失败概率的容忍大于采样失败的概率上限,
我们认为是最后得到的$h_{\mathcal{S}}$学到了有用的的东西的,
记为:
$$
|\mathcal{H}|e^{-\epsilon m}
\leq \delta
$$
借此我们可以推断出我们所需要拥有的样本数量为:
$$
m \geq \frac{\ln(|\mathcal{H}|/\delta)}{\epsilon}
$$

\textbf{
推论1.1 设$\mathcal{H}$为一个有限假设集合,
$\delta\in(0,1)$,$\epsilon>0$,当
$$
m \geq \frac{\ln(|\mathcal{H}|/\delta)}{\epsilon}
$$
成立时,从而对于任何标记函数$f$、任何分布$\mathcal{D}$,
可实现性假设最少以$1-\delta$的概率,对于每个$ERM$假设$h_{\mathcal{S}}$,
有以下不等式成立:
$$
\mathcal{L}_{\mathcal{D},f}(h_{\mathcal{S}})\leq\epsilon
$$
}

上述推论表明,对于足够大的$m$,由${ERM}_{\mathcal{H}}$规则
生成的有限假设会以$1-\delta$概率得到小于误差$\epsilon$的近似正确解,
概率在于容忍采样失败概率$\delta$,
近似在于容忍误差$\epsilon$,
即我们的算法,
可能(容忍了失败概率)会学到令我们基本满意(容忍了一定误差)的东西。
而概率近似正确,即PAC理论,将在第二章详述。

\subsection{思路总结}
\begin{enumerate}
\item 为什么不能有效学习?为什么会$\mathcal{L}_{\mathcal{D},f}(h)>\epsilon$?
\item 因为样本不具有代表性,学到的东西有问题,采样失败。
\item 怎么解决?
\item 降低采样失败概率至我们可以容忍的范围,小于$\delta$。
\item 采样失败的概率上限小于$\delta$。
\item $\sup{\mathcal{D}^m。
(\{x_i|x_i\in\mathcal{S},i=1,...m,
\mathcal{L}_{\mathcal{D},f}(h_{\mathcal{S}})>\epsilon
\})}=|\mathcal{H}|e^{-\epsilon m}\leq \delta$。
\item $m \geq \frac{\ln(|\mathcal{H}|/\delta)}{\epsilon}$
\item 当样本足够充足时,学习到的假设概率近似正确。
\end{enumerate}

\newpage


% ============================================================
%
%			第二章 Probably Approximately Correct
%
% ============================================================
\section{Probably Approximately Correct}
\subsection{PAC学习理论}
\subsubsection{PAC可学习}
接着上一章继续说,
在经验风险最小化准则下,
对于一个有限假设类,
如果有足够多的训练样本,
则我们输出的学习算法在真实世界中,
是概率近似正确的。我们做以下定义:

\textbf{
定义2.1
\textit{PAC可学习}
若存在一个函数$m_{\mathcal{H}}:{(0,1)}^2\rightarrow\mathcal{N}$
和一个学习算法,使得对于给定的$\epsilon,\delta\in(0,1)$
和任一分布$\mathcal{D}$、
任一标记函数$f:\mathcal{X}\rightarrow\{0,1\}$,
可实现假设成立时,
那么当样本数量$m\geq m_{\mathcal{H}}(\epsilon,\delta)$时,
算法将以不小于$1-\delta$的概率返回一个使
$$\mathcal{L}_{\mathcal{D},f}(h)\leq\epsilon$$
的假设$h$。
}

\subsubsection{采样复杂度}
函数$m_{\mathcal{H}}:(0,1)^2\rightarrow\mathcal{N}$
决定了假设的采样复杂度,即可以学到东西时所需要的样本数量。
采样复杂度不仅依赖于$\epsilon$和$\delta$,
同样还依赖于假设空间中的假设数量:
$$
m_{\mathcal{H}}=\frac{\ln(|\mathcal{H}|/\delta)}{\epsilon}
$$
其与假设数量的对数成正比。

\textbf{
推论2.1
当采样复杂度满足:
$$
m_{\mathcal{H}}(\epsilon,\delta)
\leq\lceil 
\frac{\ln(|\mathcal{H}|/\delta)}{\epsilon}
\rceil
$$
则任一有限类假设是PAC可学习的。
}

$m_{\mathcal{H}}(\epsilon,\delta)$被称为
假设结合$\mathcal{H}$的采样复杂度的最小函数。
由上述定义我们可知,一个假设是否是PAC可学习的,
还依赖于假设空间的大小,
而假设空间则与VC维相关,将会在后几章说明。

\subsection{泛化PAC理论}
为了让该理论更贴近实际,
则我们考虑将理论的前提约束放宽,进行泛化。
\begin{enumerate}
\item 去掉可实现性假设。
\item 考虑多分类、回归等问题。
\end{enumerate}

\subsection{不可知PAC学习}
当我们放弃可实现假设时,我们称之为不可知,
具体可以理解为:
完全一样的样本,却又不同的标记,这种情况下,
我们的算法如何学到东西。

从此处开始,为了简写,我们将
$\mathcal{D}$定义为$\mathcal{X}\times\mathcal{Y}$上的概率分布,
即将之前$\mathcal{D}$和$f$简写在一起,
作为领域集和标签集的联合概率分布,定义真实误差为:
$$
\mathcal{L}_{\mathcal{D}}(h)
\xlongequal{def}
\mathbb{P}_{(x,y)\sim\mathcal{D}}[h(x)\neq y]
\xlongequal{def}
\mathcal{D}(\{(x,y):h(x)\neq y\})
$$

在面对不可知的情况下,最好的预测器为贝叶斯预测器:
$$
f_{\mathcal{D}}(x)=
\begin{cases}
1& \mathbb{P}[y=1|x]\geq\frac{1}{2}\\
0& else
\end{cases}
$$
我们易证贝叶斯预测器是最优的:
对于每个领域集的实例$x\in\mathcal{X}$
都选择数量较多的类别,
即每个实例的分类正确概率都大于$1/2$,
没有其他分类器可以达到比它更低的错误率,对于任意预测器$g$都有
$\mathcal{L}_{\mathcal{D}}(f_{\mathcal{D}})
\leq \mathcal{L}_{\mathcal{D}}(g)$,
所以该预测器最优。
但我们并不知道实际的概率分布$\mathcal{D}$,
所以我们并没法使用这样的预测器。

所以对于不可知问题,我们只能选择容忍
(看来统计的机器学习就是一个容忍的过程Orz):

\textbf{
定义2.2 \textit{不可知PAC可学习}
若存在一个函数$m_{\mathcal{H}}:{(0,1)}^2\rightarrow\mathcal{N}$和
一个学习算法,
使得对于给定的$\epsilon,\delta\in(0,1)$
和任一分布$\mathcal{D}$,
当样本数量$m\geq m_{\mathcal{H}}(\epsilon,\delta)$时,
算法将以不小于$1-\delta$的概率返回一个使
$$\mathcal{L}_{\mathcal{D}}(h)\leq
\min_{h'\in\mathcal{H}}\mathcal{L}_{\mathcal{D}}(h')+\epsilon
$$
成立的假设$h$。
}

\subsection{学习问题建模}
此节我们讨论学习问题建模,总的来说,除了二分类问题,
学习任务还分为以下几种:
\begin{itemize}
\item 多分类
\item 回归
\end{itemize}
虽然说学习任务分为多种,但主要区别只在于损失函数不同。

\subsubsection{广义损失函数}
给定任意集合$\mathcal{H}$和定义域
$\mathcal{Z}=\mathcal{X}\times\mathcal{Y}$,令
$\ell$为$\mathcal{H}\times\mathcal{Z}$到非负实数的映射,
记为$\ell:\mathcal{H}\times\mathcal{Z}\rightarrow\mathbb{R}_{+}$,
$\ell$就是损失函数。
现在我们重新定义真实误差和经验误差:
$$
\mathcal{L}_{\mathcal{D}}(h)
\xlongequal{def}
\mathbb{E}_{z\sim\mathcal{D}}[\ell(h,z)]
$$
$$
\mathcal{L}_{\mathcal{S}}(h)
\xlongequal{def}
\frac{1}{m}\sum^m_{i=1}\ell(h,z_i)
$$
同时我们定义常用的损失函数:
\begin{itemize}
\item
0-1损失
$$
\ell_{0-1}(h,(x,y))
\xlongequal{def}
\begin{cases}
0& h(x)=y\\
1& h(x)\neq y
\end{cases}
$$
\item 平方损失
$$
\ell_{sq}(h,(x,y))
\xlongequal{def}
{(h(x)-y)}^2
$$
\end{itemize}

\textbf{
定义2.3 
\textit{广义损失函数下的不可知PAC可学习}
对于集合$\mathcal{Z}$和损失函数
$\ell:\mathcal{H}\times\mathcal{Z}\rightarrow\mathbb{R}_{+}$,
若存在一个函数$m_{\mathcal{H}}:{(0,1)}^2\rightarrow\mathcal{N}$和
一个学习算法,
使得对于给定的$\epsilon,\delta\in(0,1)$
和任一分布$\mathcal{D}$,
当样本数量$m\geq m_{\mathcal{H}}(\epsilon,\delta)$时,
算法将以不小于$1-\delta$的概率返回一个使
$$
\mathcal{L}_{\mathcal{D}}(h)\leq
\min_{h'\in\mathcal{H}}\mathcal{L}_{\mathcal{D}}(h')+\epsilon
$$
成立的假设$h$,其中
$\mathcal{L}_{\mathcal{D}}(h)=\mathbb{E}_{z\sim\mathcal{D}}[\ell(h,z)]$。
}

\subsection{思路总结}
本章主要在于泛化PAC学习的试用范围,
通过更改假设来让这个理论更具有普适性,
更贴近生活现实。
(通俗来说就是通过不断的容忍,
放缩范围,
看来搞统计的都是受啊XD)

\newpage

% ============================================================
%
%			第三章 Learning via Uniform Convergence
%
% ============================================================

\section{Learning via Uniform Convergence}
\subsection{学习的一致收敛性}
我们希望我们所得到的假设集合$\mathcal{H}$不仅仅只在训练集$\mathcal{S}$上表现良好,
同样希望它能在真实情况也有相似的表现,
既不是好一些也不是差一些,是具有相近的性能,即
$$
\forall h\in\mathcal{H}, 
\mathcal{L}_{\mathcal{S}}(h)\approx \mathcal{L}_{\mathcal{D}}(h)
$$
当我们的学习到的假设集合$\mathcal{H}$能达到上述期望时,
我们称该学习到的的假设集合$\mathcal{H}$具有一致收敛性,
同时,当经验误差和泛化误差的差异小于我们容忍限度$\epsilon$时,
我们称这时的训练样本为$\epsilon$代表性的。

\textbf{
定义3.1 
\textit{$\epsilon$代表性样本}
当训练样本满足以下不等式时
$$
\forall h\in \mathcal{H},
|\mathcal{L}_{\mathcal{S}}(h)-\mathcal{L}_{\mathcal{D}}(h)|\leq \epsilon
$$
这时训练样本$\mathcal{S}$被称为$\epsilon$代表性样本。
}

\textbf{
引理3.1 
假设一个训练集$\mathcal{S}$是$\epsilon/2$代表性的,
那么任何一个根据ERM准则输出的假设集合$\mathcal{H}$,都满足
$$
\mathcal{L}_{\mathcal{D}}(h_{\mathcal{S}})\leq
\min_{h\in\mathcal{H}}\mathcal{L}_{\mathcal{D}}(h)+\epsilon
$$
}
证明如下:对于$\forall h\in\mathcal{H}$
$$
\mathcal{L}_{\mathcal{D}}(h_{\mathcal{S}})\leq
\mathcal{L}_{\mathcal{S}}(h_{\mathcal{S}})+\frac{\epsilon}{2}\leq
\mathcal{L}_{\mathcal{S}}(h)+\frac{\epsilon}{2}\leq
\mathcal{L}_{\mathcal{D}}(h_{\mathcal{S}})+\epsilon
$$
其中第一和第三个不等式由$\epsilon/2$代表性的样本保证,第二个不等式是对假设$h$损失的放缩。

\textbf{
定义3.2 
\textit{一致收敛}
存在一个函数$m_{\mathcal{H}}^{UC}:(0,1)^2\rightarrow\mathbb{N}$,
使得对于所有$\epsilon,\delta\in(0,1)$和任意概率分布$\mathcal{D}$,
都有当$m\geq m_\mathcal{H}^{UC}$时,
至少在$1-\delta$的概率下,训练集$\mathcal{S}$是$\epsilon$代表性的,
假设集合$\mathcal{H}$具有一致收敛性。
}

\subsection{有限假设是不可知PAC可学习的}
给定$\epsilon$和$\delta$,
我们需要找一个样本大小$m$保证:
对于任何分布$\mathcal{D}$,
至少在$1-\delta$的概率下,从$\mathcal{D}$中采样得到的
独立同分布样本$\mathcal{S}=(z_1,...,z_m)$,
对于$\forall h\in\mathcal{H}$,
有$|\mathcal{L}_{\mathcal{S}}(h)-\mathcal{L}_{\mathcal{D}}(h)|\leq \epsilon$
成立,即
$$
\mathcal{D}^m(
\{\mathcal{S}:\forall h\in\mathcal{H},
|\mathcal{L}_{\mathcal{S}}(h)-\mathcal{L}_{\mathcal{D}}(h)|\leq \epsilon\}
)\geq 1-\delta
$$
根据之前的推导逐步放缩
$$
\begin{aligned}
\{\mathcal{S}:\forall h\in\mathcal{H},
|\mathcal{L}_{\mathcal{S}}(h)-\mathcal{L}_{\mathcal{D}}(h)|> \epsilon\}
 &= \bigcup_{h\in\mathcal{H}} 
\{\mathcal{S}:|\mathcal{L}_{\mathcal{S}}(h)-\mathcal{L}_{\mathcal{D}}(h)|> \epsilon \}\\
\mathcal{D}^m(
\{\mathcal{S}:\forall h\in\mathcal{H},
|\mathcal{L}_{\mathcal{S}}(h)-\mathcal{L}_{\mathcal{D}}(h)|> \epsilon\}
)&\leq
\sum_{h\in\mathcal{H}}\mathcal{D}^m(\{\mathcal{S}:\forall h\in\mathcal{H},
|\mathcal{L}_{\mathcal{S}}(h)-\mathcal{L}_{\mathcal{D}}(h)|> \epsilon\})\\
\end{aligned}
$$
根据大数定律,样本均值会随着样本数量增加,逐渐收敛至总体均值,
但是大数定律并没有明确说明有多接近,只是说明了收敛的趋势,
所以我们用Hoeffding Inequality来度量,
令$\theta_i$为随机变量$\ell(h,z_i)$,假定$\ell\in[0,1]$,有
$$
\mathcal{D}^m(\{\mathcal{S}:\forall h\in\mathcal{H},
|\mathcal{L}_{\mathcal{S}}(h)-\mathcal{L}_{\mathcal{D}}(h)|> \epsilon\})=
\mathbb{P}[|\frac{1}{m}\sum^m_{i=1}\theta_i-\mu|>\epsilon]\leq
2\exp(-2m\epsilon^2)
$$
所以
$$
\begin{aligned}
\mathcal{D}^m(
\{\mathcal{S}:\forall h\in\mathcal{H},
|\mathcal{L}_{\mathcal{S}}(h)-\mathcal{L}_{\mathcal{D}}(h)|> \epsilon\}
)&\leq
\sum_{h\in\mathcal{H}}2\exp(-2m\epsilon^2)\\
&= 2|\mathcal{H}|2\exp(-2m\epsilon^2)\\
\end{aligned}
$$
当我们令
$$
m\geq \frac{\log(2|\mathcal{H}|/\delta)}{2\epsilon^2}
$$
时就会有
$$
\mathcal{D}^m(
\{\mathcal{S}:\forall h\in\mathcal{H},
|\mathcal{L}_{\mathcal{S}}(h)-\mathcal{L}_{\mathcal{D}}(h)|> \epsilon\}
)\leq \delta
$$
\textbf{
推论3.1
$\mathcal{H}$具有一致收敛性,则其样本复杂度函数是
$$
m^{UC}_{\mathcal{H}}(\epsilon,\delta)\leq 
\lceil \frac{\log(2|\mathcal{H}|/\delta)}{2\epsilon^2} \rceil
$$
}


\newpage


% ============================================================
%
%			第四章 The Bias-Complexity Tradeoff
%
% ============================================================
\section{The Bias-Complexity Tradeoff}
\subsection{误差分解}
对于一个预测任务,
我们可以将由经验风险最小化得到的预测器误差分解为两部分:
逼近误差和估计误差,记为
$$
\mathcal{L}_{\mathcal{D}}=
\epsilon_{app}+\epsilon_{est}
$$
其中逼近误差
$$
\epsilon_{app}=\min_{h\in\mathcal{H}}
\mathcal{L}_{\mathcal{D}}(h)
$$
和估计误差
$$
\epsilon_{est}=
\mathcal{L}_{\mathcal{D}}(h_\mathcal{S})-
\min_{h\in\mathcal{H}}
\mathcal{L}_{\mathcal{D}}(h)
$$
具体解释一下就是:
逼近误差,是一个假设类集合中,
所有假设能取得最好可能结果,
此时的泛化误差;
而估计误差,
是根据经验风险最小化原则,
在有限样本上,
在假设类集合中,
能学习到的最好的假设的泛化误差。
因为在样本上学习到的最好的假设并不一定在真实情况中最好,
$h_\mathcal{S}$是我们对$\min_{h\in\mathcal{H}}
\mathcal{L}_{\mathcal{D}}(h)$中的$h$的估计。

\begin{figure}[htbp]
\centering
\includegraphics[width=4in]{1.png}
\caption{误差分解示意图}
\end{figure}

由逼近误差和估计误差的定义我们可知,
逼近误差与选择假设集合有关,
也就是选择模型种类有关,
用决策树还是对率回归来解决问题,
会产生不同的逼近误差;
而估计误差,
则与假设集合的大小、样本集的大小和复杂度相关,
样本数量越少,
假设集合中假设数量越多,
则估计误差就会越大。
在进行一个任务的时候,
如果选择丰富的假设集合,
此处的丰富可以理解为参数较多,模型较为复杂,
模型的capacity较大的模型,
则逼近误差就会较小,
模型会更容易逼近真是分布,
但同时,
参数多、复杂的模型,
会增加估计误差,
样本数量和模型复杂度都会大大增加估计误差。
此时我们就要面临一个权衡,
这被称为偏差-复杂度权衡(Bias-Complexity Tradeoff),
选择合适的模型就更为重要。

通常研究人员说设计一个模型,
就是通过减少逼近误差来减少总体的泛化误差,
我们一般做的调参,就是根据经验风险最小化准则,
来降低估计误差。

写到这里,我突然想起之前吴恩达说的,
未来是属于深度学习的言论,
他说机器学习模型的表现会随着样本量增加而变好,
但是不同模型上限不同。
随着数据量的增加,
传统算法模型capacity的上限远远低于大规模的深度学习模型,
同等数据,逼近误差深度学习模型会小得多,
但数据越来越多越来越多,
我们的估计误差也会减少,
当减少到我们足够接受的时候,
深度学习模型自然就成为机器学习的未来了。

\begin{figure}[htbp]
\centering
\includegraphics[width=4in]{2.png}
\caption{模型表现示意图}
\end{figure}


\subsection{偏差方差分解}
此处我再写点周志华老师书上写的内容吧,
也是对同一个问题的不同表述和解释。
我们对学习算法的期望预测为
$$
\bar{f}(x)=\mathbb{E}_{\mathcal{D}}[f(x;\mathcal{D})]
$$
使用样本数相同的不同训练集产生的方差为
$$
var(x)=\mathbb{E}_{\mathcal{D}}[(f(x;\mathcal{D})-\bar{f}(x))^2]
$$
噪声为
$$
\epsilon^2=\mathbb{E}_{\mathcal{D}}[(y_{\mathcal{D}}-y)^2]
$$
期望输出与真实标记的差别为
$$
bias^2(x)=(\bar{f}(x)-y)^2
$$
进行误差分解之后我们有
$$
\mathbb{E}(f;\mathcal{D})=
bias^2(x)+var(x)+\epsilon^2
$$
具体如下图所示

\begin{figure}[htbp]
\centering
\includegraphics[width=4in]{3.png}
\caption{偏差-方差分解}
\end{figure}

所以周老师课上这样总结的,
模型的泛化能力由这几方面决定:
\begin{enumerate}
\item
学习算法本身的能力$bias$,
也就是逼近误差;
\item
数据的充分性$var$,
也就是估计误差;
\item
学习任务本身难度$\epsilon^2$,
数据有限,
真理只能接近,却不能到达,
此处的平方仅表示数值上大于$0$。
\end{enumerate}

具体的分解步骤大家可以看周老师的西瓜书p44。
















\newpage




% ============================================================
%
%			第五章 The VC-Dimension
%
% ============================================================
\section{The VC-Dimension}


\newpage



% ============================================================
%
%			附录1 不等式证明
%
% ============================================================

\section{附录1:不等式证明}
\subsection{Markov Inequality}
马尔科夫不等式:
$$
\mathbb{P}(X\geq a)\leq \frac{\mathbb{E}(Z)}{a}
$$
其中$X$是非负的随机变量。

证:
$$
\begin{aligned}
\mathbb{E}(X) &=  \int_0^{+\infty}{xf(x)dx} \\
	&= \int_0^{a}{xf(x)dx} + \int_a^{+\infty}{xf(x)dx} \\
	&\geq 
	\int_0^{a}{0f(x)dx} + \int_a^{+\infty}{af(x)dx}\\
	&\geq 0 + \int_a^{+\infty}{af(x)dx} 
   = a\int_a^{+\infty}{f(x)dx} = a\mathbb{P}(X>a)
\end{aligned}
$$
移动一下$a$的位置,不等式得证,
其中第二行到第三行是将积分中的$x$换成积分下限。

\subsection{引理1}

设$Z$是一个取值$[0,1]$的随机变量,假定$\mathbb{E}[Z]=\mu$,
那么对于任意$a\in (0,1)$,都有
$$
\mathbb{P}[Z>1-a]\geq\frac{\mu-(1-a)}{a}
$$

证:令$Y=1-Z$,则$Y$是非负随机变量,且$\mathbb{Y}=1-\mathbb{Z}=1-\mu$,
根据马尔科夫不等式
$$
\mathbb{P}[Z\leq 1-a]=
\mathbb{P}[1-Z\geq a]=
\mathbb{P}[Y\geq a]\leq \frac{\mathbb{E}[Y]}{a}=\frac{1-\mu}{a}
$$
所以
$$
\mathbb{P}[Z>1-a]\geq 1-\frac{1-\mu}{a} = \frac{a+\mu-1}{a}
$$


\subsection{Chebyshev Inequality}
切比雪夫不等式:
$$
\forall a>0,
\mathbb{P}[|Z-\mathbb{E}[Z]|\geq a] = 
\mathbb{P}[(Z-\mathbb{E}[Z])^2 \geq a^2] \leq
\frac{Var[Z]}{a^2}
$$
其中$Var[Z]=\mathbb{E}[(Z-\mathbb{E}[Z])^2]$是$Z$的方差。

证明:令$Y=(Z-\mathbb{E}[Z])^2$带入马尔科夫不等式即可得证。

\subsection{引理2}
设$Z_1,...,Z_m$是独立同分布的随机变量,
假定$\mathbb{E}[Z]=\mu$且$Var[Z]\leq 1$,
那么对于任意的$\delta\in(0,1)$有
$$
|\frac{1}{m}\sum^m_{i=1}Z_i-\mu|\leq \sqrt{\frac{1}{\delta m}}
$$
成立的概率大于$1-\delta$。

证:由切比雪夫不等式,对于$a>0$,我们有
$$
\mathbb{P}[|\frac{1}{m}\sum^m_{i=1}Z_i-\mu|>a]\leq 
\frac{Var[Z_1]}{ma^2}\leq \frac{1}{ma^2}
$$
令等式右边等于$\delta$即可得证。


\subsection{Chernoff Bound}
切尔诺夫界:
假设$Z_1,...,Z_m$是独立的伯努利变量,
其中任意的$i$都有
$\mathbb{P}[Z_i=1]=p_i$。
令$p=\sum^m_{i=1}p_i$和
$Z=\sum^m_{i=1}Z_i$,
对于$t>0$有
$$
\mathbb{P}[Z>(1+\delta)p]=
\mathbb{P}[e^{tZ}>e^{t(1+\delta)p}]\leq
\frac{\mathbb{E}[e^{tZ}]}{e^{t(1+\delta)p}}
$$
其中,第一个等号成立是因为$e^x$单调递增,
之后的不等关系成立是因为马尔科夫不等式。

\subsection{引理3}
根据之前的切尔诺夫界,
对于$\mathbb{E}[e^{tZ}]$有
$$
\begin{aligned}
\mathbb{E}[e^{tZ}] &= \mathbb{E}[e^{t\sum_iZ_i}]= \mathbb{E}[\prod_ie^{tZ_i}]\\
	&= \prod_i\mathbb{E}[e^{tZ_i}]\\
	&= \prod_i(p_ie^{(1\times t)}+(1-p_i)e^{0\times t})\\
	&= \prod_i(1+p_i(e^t-1))\\
	&\leq \prod_ie^{p_i(e^t-1)}\\
	&= e^{\sum_ip_i(e^t-1)}\\
	&= e^{p(e^t-1)}
\end{aligned}
$$
其中不等式成立因为$e^x\approx 1+x+\sum^\infty_{n=2}\frac{x^n}{n!}$,
根据不同情况更改$t$的值可以得到不同的概率上界。

取$t=\log(1+\delta)$,则
$$
\mathbb{P}[Z>(1+\delta)p]\leq e^{-h(\delta)p}
$$
其中
$$
h(\delta)=(1+\delta)\log(1+\delta)-\delta
$$
根据$h(\delta)\geq \frac{\delta^2}{(2+2\delta/3)}$有
$$
\mathbb{P}[Z>(1+\delta)p]\leq e^{-p\frac{\delta^2}{2+2a/3}}
$$

取$t=-log(1-\delta)$则有
$$
\mathbb{P}[Z<(1-\delta)p]\leq
\frac{e^{-\delta p}}{e^{(1-\delta)\log(1-\delta)p}}=
e^{-ph(-\delta)}
$$
根据$h(-\delta)\geq h(\delta)$有
$$
\mathbb{P}[Z<(1-\delta)p]\leq
e^{-ph(\delta)}\leq e^{ph(\delta)}\leq
e^{-p\frac{\delta^2}{2+2a/3}}
$$


\subsection{Hoeffding Inequality}
霍夫丁不定式:
假设$Z_1,...,Z_m$是独立同分布的随机变量,
令$\overline{Z}=\frac{1}{m}\sum^m_{i=1}Z_i$,
假定$\mathbb{E}[\bar{Z}]=\mu$且
$\mathbb{P}[a\leq Z_i \leq b]=1$对于所有$i$成立,
那么对于任意$\epsilon$有
$$
\mathbb{P}[|\frac{1}{m}\sum^m_{i=1}Z_i-\mu|>\epsilon]\leq
2e^{(-\frac{-2m\epsilon^2}{(b-a)^2})}
$$

证:
记$X_i=Z_i-\mathbb{E}[Z_i]$且$\bar{X}$,
且$\bar{X}=\frac{1}{m}\sum^m_{i=1}X_i$,
对于任意的$\lambda, \epsilon > 0$有
$$
\mathbb{P}[\bar{X}-\epsilon]=
\mathbb{P}[e^{-\lambda\bar{X}}-e^{-\lambda\epsilon}]\leq
e^{-\lambda\epsilon}\mathbb{E}[e^{-\lambda\bar{X}}]
$$
其中
$$
\mathbb{E}[e^{-\lambda\bar{X}}]=
\mathbb{E}[\prod_i e^{-\lambda X_i/m}]=
\prod_i\mathbb{E}[e^{-\lambda X_i/m}]
$$

对于凸函数$f(x)=e^x$,
对于任意$\alpha \in [0,1]$和$x\in[a,b]$都有
满足
$$f(x)\leq \alpha f(a) + (1-\alpha) f(b)
$$
令$\alpha=\frac{b-x}{b-a}\in[0,1]$,则
$$
e^{\lambda x}\leq 
\frac{b-x}{b-a}e^{\lambda a}+\frac{x-a}{b-a}e^{\lambda b}
$$
当$\mathbb{E}[X]=0$时,对两边同时取期望
$$
\mathbb{E}[e^{\lambda x}]\leq
\frac{b-\mathbb{E}[X]}{b-a}e^{\lambda a}+\frac{\mathbb{E}[X]-a}{b-a}e^{\lambda b}=
\frac{b}{b-1}e^{\lambda a} - \frac{b}{b-1}e^{\lambda b} 
$$
记$h=\lambda(b-a)$和$p=\frac{-a}{b-a}$,
可以将上式右边写作$e^{-hp+\log(1-p+pe^h)}$,记为$e^{L(h)}$。
我们将$L(h)$作泰勒展开
$$
L(h)=\frac{L(0)}{0!}(x-0)^0+\frac{L'(0)}{1!}(x-0)^1+\frac{L''(0)}{2!}(x-0)^2+o(x^2)
$$
易得$L(0)=L'(0)=0$,而
$$
L''(h)=\frac{(1-p)pe^h}{(1-p+pe^h)^2}
$$
得$L''(0)=(1-p)p\leq 1/4$,将其放缩后得到
$$
L(h)=\frac{L''(0)}{2!}(x-0)^2+o(x^2)\leq
\frac{1/4}{2!}(x-0)^2=
\frac{h^2}{8}
$$
将其回带,对于任意$i$有
$$
\mathbb{E}[e^{\lambda X_i/m}]\leq e^{\frac{\lambda^2(b-a)^2}{8m^2}}
$$
因此
$$
\mathbb{E}[\bar{X}\geq \epsilon]\leq
e^{-\lambda\epsilon}\prod_ie^{\frac{\lambda^2(b-a)^2}{8m^2}}=
e^{-\lambda\epsilon+\frac{\lambda^2(b-a)^2}{8m}}
$$
此时我们令$\lambda=4m\epsilon/(b-a)^2$
$$
\mathbb{E}[\bar{X}\geq \epsilon]\leq
e^{-4m\epsilon/(b-a)^2\epsilon+\frac{(4m\epsilon/(b-a)^2)^2(b-a)^2}{8m}}
$$
$$
\mathbb{E}[\bar{X}\geq \epsilon]\leq
e^{-\frac{2m\epsilon^2}{(b-a)^2}}
$$
将$\bar{X}_i=Z_i-\mathbb{E}[Z_i]$带回,不等式得证。





%
%$$
%L(h)=-hp+\log(1-p+pe^h)\leq\frac{h^2}{8}
%$$










































\end{document}

\documentclass[UTF8]{ctexart}
% use package here
\usepackage{mathrsfs}
\usepackage{extarrows}
\usepackage{amsmath}
\DeclareMathOperator*{\argmax}{arg\,max}
\usepackage{enumerate}




% information of front page 
\title{Understanding Machine Learning}
\author{Yuyang Zhang}
\date{\today}

% body
\begin{document}
\maketitle
\newpage

\tableofcontents
\newpage


\section{why can machine learn}
\subsection{基本符号}
\subsubsection{输入}
Domain Set:
一个任意集合$\mathcal{X}$,
也作领域集。
可以理解为所有样本的集合,
其中每个样本通常以一个能够表征其特征的向量表示。

Label Set:
标签集$\mathcal{Y}$。
样本所属于的类别,
通常二分类问题,
标签集为$\{0,1\}$或者是$\{-1,+1\}$。

Training Data:
训练数据$\mathcal{S}=\{(x_1,y_1),...,(x_m,y_m)\}$,
也叫训练集,样本集。

\subsubsection{输出}
Predicting Rule:
预测规则,$h:\mathcal{X}\rightarrow\mathcal{Y}$。
该规则是一个由样本集合到标签集的映射,
可以理解为预测器(predictor),
\textbf	{假设(hypothesis)},
分类器(classifier),等等。

\subsubsection{数据生成模型}
我们假定样本$\mathcal{S}$是由概率分布$\mathcal{D}$生成,
并且根据一个标记函数$f:\mathcal{X}\rightarrow\mathcal{Y}$,
来标记样本的类别,
对于任意$i=1,...,m$都有
$y_i=f(x_i)$,
然而我们并不知道概率分布$\mathcal{D}$与标记函数$f$,
我们的目标就是找一个合适的假设$h$,
令其与标记函数$f$可以对样本做出相同的标记。

\subsubsection{衡量标准}
我们定义分类误差为:
未能成功预测随机数据点正确标签的概率,
即对于随机的一个$x\in\mathcal{X}$,
$h(x)\neq f(x)$的概率。


定义$\mathcal{A}\subseteq\mathcal{X}$为一个领域子集,
$\mathcal{A}$中的任意实例$x\in\mathcal{A}$的出现概率由
$\mathcal{D}(\mathcal{A})$所决定。
通常,
我们称$\mathcal{A}$为一个事件,
$\mathcal{A}=\{x\in\mathcal{X}:\pi(x)=1\}$,
其中$\pi:\mathcal{X}\rightarrow\{0,1\}$,
表示样本是否被观测到。
我们也将$\mathcal{D}(\mathcal{A})$写作
$\mathcal{P}_{x\sim\mathcal{D}} [\pi(x)]$。
此时我们可以定义假设$h$的错误率为:
$$
\mathcal{L}_{\mathcal{D},f}(h)\xlongequal{def}
\mathcal{P}_{x\sim\mathcal{D}} [h(x)\neq f(x)]\xlongequal{def}
\mathcal{D}(\{x|h(x)\neq f(x)\})
$$
其中误差的测量是基于概率分布$\mathcal{D}$和标记函数$f$的,
$\mathcal{L}_{\mathcal{D},f}(h)$也称为泛化误差,
\textbf{损失}
或者$h$的真实误差。

\subsection{经验风险最小化}
机器学习的过程都是基于训练集$\mathcal{S}$的,
训练集$\mathcal{S}$由未知分布$\mathcal{D}$从
领域集$\mathcal{X}$中采样得出,并由标记函数$f$标记,
机器学习的输出是一个基于训练集$\mathcal{S}$的假设,
$h_{\mathcal{S}}:\mathcal{X}\rightarrow\mathcal{Y}$。

由于我们并不知道分布$\mathcal{D}$与标记函数$f$,
所以我们只能根据训练集$\mathcal{S}$来判断
我们所选择假设的表现,定义训练误差为:
$$
\mathcal{L}_{\mathcal{S}}\xlongequal{def}
\frac{|\{x_i|h(x_i)\neq y_i,i=1,..,m\}|}{m}
$$
训练误差也称作经验误差和经验风险。

因为训练集$\mathcal{S}$是领域$\mathcal{X}$的一个子集,
所以训练样本是真实世界的一个缩影,
正如概率统计的一个核心思想,
通过样本反映总体。
所以我们认为利用样本集寻找一个较好的假设是可行的,
即最小化训练误差$\mathcal{L}_{\mathcal{S}}(h)$,
这称之为经验风险最小化,
ERM(Experience Risk Minimize)。

\subsubsection{过拟合}
一个假设在训练集上效果优异,
但在真实世界中表现却糟糕,
这种现象称之为过拟合。
当我们过度追求经验风险最小化原则时,
我们就有可能面临过拟合的风险。

\subsection{归纳偏好}
虽然经验风险最小化会有过拟合的风险,
相比于抛弃这个原则,我们更愿意去修正这个原则,
考虑我们对于假设的归纳偏好。
我们通常的解决方案是根据我们定好的归纳偏好,
在有限的假设空间中去搜索所要用的假设,
这些假设的集合成为假设类,记为$\mathcal{H}$,
则我们的学习过程可以记为:
$$
{ERM}_{\mathcal{H}}(\mathcal{S})\in 
\argmax_{h\in\mathcal{H}}
\mathcal{L}_{\mathcal{S}}(h)
$$
我们常见的正则化,就是一种归纳偏好,
$L1$正则化表明我们的归纳偏好是更喜欢参数稀疏的假设。

\subsection{为什么可以学习到东西}
本节旨在说明:
当拥有足够多样本时,
在有限假设空间$\mathcal{H}$中,
经验风险最小化${ERM}_{\mathcal{H}}$原则不会出现过拟合,
即我们通过训练样本,
可以找到一个足够好的假设$h_s$,
在真实世界中表现也足够好。


\textbf{定义1 \textit{可实现性假设}
存在$h^*\in\mathcal{H}$,
使得$\mathcal{L}_{\mathcal{D},f}(h^*)=0$。}:


该假设意味着,对于随机样本集$\mathcal{S}$,
由概率分布$\mathcal{D}$采样,
由标记函数$f$标记,
以概率$1$使得$L_{\mathcal{S}}(h^*)=0$,
其中样本集$\mathcal{S}$中的样本是独立同分布的。
我们定义$h_{\mathcal{S}}$为对$\mathcal{S}$
利用${ERM}_{\mathcal{H}}$得到的结果:
$$
h_{\mathcal{S}}\in 
\argmax_{h\in\mathcal{H}}
\mathcal{L}_{\mathcal{S}}(h)
$$

因为样本集$\mathcal{S}$仍是
领域集$\mathcal{X}$的子集,
是根据分布随机得到的实例集合,
会有一定概率使得采样得到的样本不具有代表性,
并不能反映真实的总体情况,
所以根据样本集$\mathcal{S}$得到的假设$h_{\mathcal{S}}$
并不一定准确,
在真实世界中的表现有可能很差。
因此,我们选择一定程度的容忍,
容忍会有一定几率采样到不具有代表性的样本,
一般来说,我们定义采样得到不具有代表性样本的概率之多为$\delta$,
则$1-\delta$为置信参数。
同时对于假设的预测效果,我们能容忍的损失上限为$\epsilon$,
称为精度参数,如果
$\mathcal{L}_{\mathcal{D},f}(h_{\mathcal{S}})>\epsilon$,
那么这是一个差的假设,
反之,则是一个好的假设。

对于我们的样本集$\mathcal{S}$,最好的假设$h_{\mathcal{S}}$仍有
$\mathcal{L}_{\mathcal{D},f}(h_{\mathcal{S}})>\epsilon$,
则这组样本采样失败,
因为我们的${ERM}_{\mathcal{H}}$无法从$\mathcal{S}$中
学到有用的东西。
当样本集中所有样本$(x_1,...,x_m)$,
都不具有代表性,我们记为:
$$
\{x_i|x_i\in\mathcal{S},i=1,...m,
\mathcal{L}_{\mathcal{D},f}(h_{\mathcal{S}})>\epsilon
\}
$$
则采样失败的概率上界为:
$$
\mathcal{D}^m
(
\{x_i|x_i\in\mathcal{S},i=1,...m,
\mathcal{L}_{\mathcal{D},f}(h_{\mathcal{S}})>\epsilon
\}
)
$$
因为每个样本都是不具有代表性的样本,
所以是采样失败的概率的上界。
设$\mathcal{H}_B$为差的假设的集合:
$$
\mathcal{H}_B=
\{
h|\mathcal{L}_{\mathcal{D},f}(h)>\epsilon,
h\in\mathcal{H}
\}
$$
同时设:
$$
M=\{x|x\in\mathcal{S},
\exists h \in \mathcal{H}_B,
\mathcal{L}_{\mathcal{S}}(h)=0
\}
$$
为误导集,误导集使差的假设在训练样本$\mathcal{S}$
上表现良好,但在真实世界中表现较差。
因为有可实现假设
$$
\exists h^*,\mathcal{L}_{\mathcal{D},f}(h^*)=0
$$
且有
$$
h_{\mathcal{S}}\in 
\argmax_{h\in\mathcal{H}}
\mathcal{L}_{\mathcal{S}}(h)
$$
产生$\mathcal{L}_{\mathcal{D},f}(h_{\mathcal{S}})>\epsilon$
是因为样本集不好,
样本集采样失败,
所以,
当且仅当$\mathcal{S}\subseteq M$时,
才会出现$\mathcal{L}_{\mathcal{D},f}(h_{\mathcal{S}})>\epsilon$,
我们将其表示为:
$$
\{x_i|x_i\in\mathcal{S},i=1,...m,
\mathcal{L}_{\mathcal{D},f}(h_{\mathcal{S}})>\epsilon
\}\subseteq M
$$
所以我们有
$$
\mathcal{D}^m
(\{x_i|x_i\in\mathcal{S},i=1,...m,
\mathcal{L}_{\mathcal{D},f}(h_{\mathcal{S}})>\epsilon
\})
\leq \mathcal{D}^m(M)
$$
而$M$又可以写作:
$$
M=\bigcup_{h\in\mathcal{H}_B}
\{
x|x\in \mathcal{S},
\mathcal{L}_{\mathcal{S}}(h)=0
\}
$$
则:
$$
\mathcal{D}^m
(\{x_i|x_i\in\mathcal{S},i=1,...m,
\mathcal{L}_{\mathcal{D},f}(h_{\mathcal{S}})>\epsilon
\})
\leq \mathcal{D}^m(\bigcup_{h\in\mathcal{H}_B}
\{
x|x\in \mathcal{S},
\mathcal{L}_{\mathcal{S}}(h)=0
\})
$$
由$P(A\cup B)\leq P(A)+P(B)$得:
$$
\mathcal{D}^m(\bigcup_{h\in\mathcal{H}_B}
\{
x|x\in \mathcal{S},
\mathcal{L}_{\mathcal{S}}(h)=0
\})
\leq
\sum_{h\in\mathcal{H}_B}
\mathcal{D}^m(\{
x|x\in \mathcal{S},
\mathcal{L}_{\mathcal{S}}(h)=0
\})
$$
其中我们将$\mathcal{S}$拆开,
$$
\mathcal{D}^m(\{
x|x\in \mathcal{S},
\mathcal{L}_{\mathcal{S}}(h)=0
\}
=\mathcal{D}^m(\{
x_i|h(x_i)=f(x_i),
x_i\in \mathcal{S},
i=1,...m
\})
$$
$$
\mathcal{D}^m(\{
x|x\in \mathcal{S},
\mathcal{L}_{\mathcal{S}}(h)=0
\}
=\prod^m_{i=1}\mathcal{D}(\{
x_i|h(x_i)=f(x_i),
x_i\in\mathcal{S}
\})
$$
根据$1-\epsilon\leq e^{-\epsilon}$,
对于等号右边的连乘的每一项都有:
$$
\mathcal{D}(\{
x_i|h(x_i)=y_i,
x_i\in\mathcal{S}
\})
=1-\mathcal{L}_{\mathcal{D},f}(h)
\leq 1-\epsilon
$$
对于所有的$h\in\mathcal{H}_B$:
$$
\mathcal{D}^m(\{
x|x\in \mathcal{S},
\mathcal{L}_{\mathcal{S}}(h)=0
\}
\leq (1-\epsilon)^m
\leq e^{-\epsilon m}
$$
可得:
$$
\mathcal{D}^m
(\{x_i|x_i\in\mathcal{S},i=1,...m,
\mathcal{L}_{\mathcal{D},f}(h_{\mathcal{S}})>\epsilon
\})
\leq |\mathcal{H}_B|e^{-\epsilon m}
\leq |\mathcal{H}|e^{-\epsilon m}
$$
所以,若我们对采样失败概率的容忍大于采样失败的概率上限,
我们认为是最后得到的$h_{\mathcal{S}}$学到了有用的的东西的,
记为:
$$
|\mathcal{H}|e^{-\epsilon m}
\leq \delta
$$
借此我们可以推断出我们所需要拥有的样本数量为:
$$
m \geq \frac{\ln(|\mathcal{H}|/\delta)}{\epsilon}
$$

\textbf{
推论1 设$\mathcal{H}$为一个有限假设集合,
$\delta\in(0,1)$,$\epsilon>0$,当
$$
m \geq \frac{\ln(|\mathcal{H}|/\delta)}{\epsilon}
$$
成立时,从而对于任何标记函数$f$、任何分布$\mathcal{D}$,
可实现性假设最少以$1-\delta$的概率,对于每个$ERM$假设$h_{\mathcal{S}}$,
有以下不等式成立:
$$
\mathcal{L}_{\mathcal{D},f}(h_{\mathcal{S}})\leq\epsilon
$$
}

上述推论表明,对于足够大的$m$,由${ERM}_{\mathcal{H}}$规则
生成的有限假设会以$1-\delta$概率得到小于误差$\epsilon$的近似正确解,
概率在于容忍采样失败概率$\delta$,
近似在于容忍误差$\epsilon$,
即我们的算法,
可能(容忍了失败概率)会学到令我们基本满意(容忍了一定误差)的东西。
而概率近似正确,即PAC理论,将在第二章详述。

\subsection{思路总结}
\begin{enumerate}
\item 为什么不能有效学习?为什么会$\mathcal{L}_{\mathcal{D},f}(h)>\epsilon$?
\item 因为样本不具有代表性,学到的东西有问题,采样失败。
\item 怎么解决?
\item 降低采样失败概率至我们可以容忍的范围,小于$\delta$。
\item 采样失败的概率上限小于$\delta$。
\item $\sup{\mathcal{D}^m
(\{x_i|x_i\in\mathcal{S},i=1,...m,
\mathcal{L}_{\mathcal{D},f}(h_{\mathcal{S}})>\epsilon
\})}=|\mathcal{H}|e^{-\epsilon m}\leq \delta$
\item $m \geq \frac{\ln(|\mathcal{H}|/\delta)}{\epsilon}$
\item 当样本足够充足时,学习到的假设概率近似正确。
\end{enumerate}













\end{document}

\documentclass[UTF8]{ctexart}
% use package here
\usepackage{mathrsfs}
\usepackage{extarrows}
\usepackage{amsmath}
\usepackage{amssymb}
\DeclareMathOperator*{\argmax}{arg\,max}
\usepackage{enumerate}




% information of front page 
\title{Understanding Machine Learning}
\author{Yuyang Zhang}
\date{\today}

% body
\begin{document}
\maketitle
\newpage

\tableofcontents
\newpage



% ============================================================
%
%			第一章 Why can machine learn
%
% ============================================================

\section{why can machine learn}
\subsection{基本符号}
\subsubsection{输入}
Domain Set:
一个任意集合$\mathcal{X}$,
也作领域集。
可以理解为所有样本的集合,
其中每个样本通常以一个能够表征其特征的向量表示。

Label Set:
标签集$\mathcal{Y}$。
样本所属于的类别,
通常二分类问题,
标签集为$\{0,1\}$或者是$\{-1,+1\}$。

Training Data:
训练数据$\mathcal{S}=\{(x_1,y_1),...,(x_m,y_m)\}$,
也叫训练集,样本集。

\subsubsection{输出}
Predicting Rule:
预测规则,$h:\mathcal{X}\rightarrow\mathcal{Y}$。
该规则是一个由样本集合到标签集的映射,
可以理解为预测器(predictor),
\textbf	{假设(hypothesis)},
分类器(classifier),等等。

\subsubsection{数据生成模型}
我们假定样本$\mathcal{S}$是由概率分布$\mathcal{D}$生成,
并且根据一个标记函数$f:\mathcal{X}\rightarrow\mathcal{Y}$,
来标记样本的类别,
对于任意$i=1,...,m$都有
$y_i=f(x_i)$,
然而我们并不知道概率分布$\mathcal{D}$与标记函数$f$,
我们的目标就是找一个合适的假设$h$,
令其与标记函数$f$可以对样本做出相同的标记。

\subsubsection{衡量标准}
我们定义分类误差为:
未能成功预测随机数据点正确标签的概率,
即对于随机的一个$x\in\mathcal{X}$,
$h(x)\neq f(x)$的概率。


定义$\mathcal{A}\subseteq\mathcal{X}$为一个领域子集,
$\mathcal{A}$中的任意实例$x\in\mathcal{A}$的出现概率由
$\mathcal{D}(\mathcal{A})$所决定。
通常,
我们称$\mathcal{A}$为一个事件,
$\mathcal{A}=\{x\in\mathcal{X}:\pi(x)=1\}$,
其中$\pi:\mathcal{X}\rightarrow\{0,1\}$,
表示样本是否被观测到。
我们也将$\mathcal{D}(\mathcal{A})$写作
$\mathbb{P}_{x\sim\mathcal{D}} [\pi(x)]$。
此时我们可以定义假设$h$的错误率为:
$$
\mathcal{L}_{\mathcal{D},f}(h)\xlongequal{def}
\mathbb{P}_{x\sim\mathcal{D}} [h(x)\neq f(x)]\xlongequal{def}
\mathcal{D}(\{x|h(x)\neq f(x)\})
$$
其中误差的测量是基于概率分布$\mathcal{D}$和标记函数$f$的,
$\mathcal{L}_{\mathcal{D},f}(h)$也称为泛化误差,
\textbf{损失}
或者$h$的真实误差。

\subsection{经验风险最小化}
机器学习的过程都是基于训练集$\mathcal{S}$的,
训练集$\mathcal{S}$由未知分布$\mathcal{D}$从
领域集$\mathcal{X}$中采样得出,并由标记函数$f$标记,
机器学习的输出是一个基于训练集$\mathcal{S}$的假设,
$h_{\mathcal{S}}:\mathcal{X}\rightarrow\mathcal{Y}$。

由于我们并不知道分布$\mathcal{D}$与标记函数$f$,
所以我们只能根据训练集$\mathcal{S}$来判断
我们所选择假设的表现,定义训练误差为:
$$
\mathcal{L}_{\mathcal{S}}\xlongequal{def}
\frac{|\{x_i|h(x_i)\neq y_i,i=1,..,m\}|}{m}
$$
训练误差也称作经验误差和经验风险。

因为训练集$\mathcal{S}$是领域$\mathcal{X}$的一个子集,
所以训练样本是真实世界的一个缩影,
正如概率统计的一个核心思想,
通过样本反映总体。
所以我们认为利用样本集寻找一个较好的假设是可行的,
即最小化训练误差$\mathcal{L}_{\mathcal{S}}(h)$,
这称之为经验风险最小化,
ERM(Experience Risk Minimize)。

\subsubsection{过拟合}
一个假设在训练集上效果优异,
但在真实世界中表现却糟糕,
这种现象称之为过拟合。
当我们过度追求经验风险最小化原则时,
我们就有可能面临过拟合的风险。

\subsection{归纳偏好}
虽然经验风险最小化会有过拟合的风险,
相比于抛弃这个原则,我们更愿意去修正这个原则,
考虑我们对于假设的归纳偏好。
我们通常的解决方案是根据我们定好的归纳偏好,
在有限的假设空间中去搜索所要用的假设,
这些假设的集合成为假设类,记为$\mathcal{H}$,
则我们的学习过程可以记为:
$$
{ERM}_{\mathcal{H}}(\mathcal{S})\in 
\argmax_{h\in\mathcal{H}}
\mathcal{L}_{\mathcal{S}}(h)
$$
我们常见的正则化,就是一种归纳偏好,
$L1$正则化表明我们的归纳偏好是更喜欢参数稀疏的假设。

\subsection{为什么可以学习到东西}
本节旨在说明:
当拥有足够多样本时,
在有限假设空间$\mathcal{H}$中,
经验风险最小化${ERM}_{\mathcal{H}}$原则不会出现过拟合,
即我们通过训练样本,
可以找到一个足够好的假设$h_s$,
在真实世界中表现也足够好。


\textbf{定义1.1 \textit{可实现性假设}
存在$h^*\in\mathcal{H}$,
使得$\mathcal{L}_{\mathcal{D},f}(h^*)=0$。}:


该假设意味着,对于随机样本集$\mathcal{S}$,
由概率分布$\mathcal{D}$采样,
由标记函数$f$标记,
以概率$1$使得$L_{\mathcal{S}}(h^*)=0$,
其中样本集$\mathcal{S}$中的样本是独立同分布的。
我们定义$h_{\mathcal{S}}$为对$\mathcal{S}$
利用${ERM}_{\mathcal{H}}$得到的结果:
$$
h_{\mathcal{S}}\in 
\argmax_{h\in\mathcal{H}}
\mathcal{L}_{\mathcal{S}}(h)
$$

因为样本集$\mathcal{S}$仍是
领域集$\mathcal{X}$的子集,
是根据分布随机得到的实例集合,
会有一定概率使得采样得到的样本不具有代表性,
并不能反映真实的总体情况,
所以根据样本集$\mathcal{S}$得到的假设$h_{\mathcal{S}}$
并不一定准确,
在真实世界中的表现有可能很差。
因此,我们选择一定程度的容忍,
容忍会有一定几率采样到不具有代表性的样本,
一般来说,我们定义采样得到不具有代表性样本的概率之多为$\delta$,
则$1-\delta$为置信参数。
同时对于假设的预测效果,我们能容忍的损失上限为$\epsilon$,
称为精度参数,如果
$\mathcal{L}_{\mathcal{D},f}(h_{\mathcal{S}})>\epsilon$,
那么这是一个差的假设,
反之,则是一个好的假设。

对于我们的样本集$\mathcal{S}$,最好的假设$h_{\mathcal{S}}$仍有
$\mathcal{L}_{\mathcal{D},f}(h_{\mathcal{S}})>\epsilon$,
则这组样本采样失败,
因为我们的${ERM}_{\mathcal{H}}$无法从$\mathcal{S}$中
学到有用的东西。
当样本集中所有样本$(x_1,...,x_m)$,
都不具有代表性,我们记为:
$$
\{x_i|x_i\in\mathcal{S},i=1,...m,
\mathcal{L}_{\mathcal{D},f}(h_{\mathcal{S}})>\epsilon
\}
$$
则采样失败的概率上界为:
$$
\mathcal{D}^m
(
\{x_i|x_i\in\mathcal{S},i=1,...m,
\mathcal{L}_{\mathcal{D},f}(h_{\mathcal{S}})>\epsilon
\}
)
$$
因为每个样本都是不具有代表性的样本,
所以是采样失败的概率的上界。
设$\mathcal{H}_B$为差的假设的集合:
$$
\mathcal{H}_B=
\{
h|\mathcal{L}_{\mathcal{D},f}(h)>\epsilon,
h\in\mathcal{H}
\}
$$
同时设:
$$
M=\{x|x\in\mathcal{S},
\exists h \in \mathcal{H}_B,
\mathcal{L}_{\mathcal{S}}(h)=0
\}
$$
为误导集,误导集使差的假设在训练样本$\mathcal{S}$
上表现良好,但在真实世界中表现较差。
因为有可实现假设
$$
\exists h^*,\mathcal{L}_{\mathcal{D},f}(h^*)=0
$$
且有
$$
h_{\mathcal{S}}\in 
\argmax_{h\in\mathcal{H}}
\mathcal{L}_{\mathcal{S}}(h)
$$
产生$\mathcal{L}_{\mathcal{D},f}(h_{\mathcal{S}})>\epsilon$
是因为样本集不好,
样本集采样失败,
所以,
当且仅当$\mathcal{S}\subseteq M$时,
才会出现$\mathcal{L}_{\mathcal{D},f}(h_{\mathcal{S}})>\epsilon$,
我们将其表示为:
$$
\{x_i|x_i\in\mathcal{S},i=1,...m,
\mathcal{L}_{\mathcal{D},f}(h_{\mathcal{S}})>\epsilon
\}\subseteq M
$$
所以我们有
$$
\mathcal{D}^m
(\{x_i|x_i\in\mathcal{S},i=1,...m,
\mathcal{L}_{\mathcal{D},f}(h_{\mathcal{S}})>\epsilon
\})
\leq \mathcal{D}^m(M)
$$
而$M$又可以写作:
$$
M=\bigcup_{h\in\mathcal{H}_B}
\{
x|x\in \mathcal{S},
\mathcal{L}_{\mathcal{S}}(h)=0
\}
$$
则:
$$
\mathcal{D}^m
(\{x_i|x_i\in\mathcal{S},i=1,...m,
\mathcal{L}_{\mathcal{D},f}(h_{\mathcal{S}})>\epsilon
\})
\leq \mathcal{D}^m(\bigcup_{h\in\mathcal{H}_B}
\{
x|x\in \mathcal{S},
\mathcal{L}_{\mathcal{S}}(h)=0
\})
$$
由$P(A\cup B)\leq P(A)+P(B)$得:
$$
\mathcal{D}^m(\bigcup_{h\in\mathcal{H}_B}
\{
x|x\in \mathcal{S},
\mathcal{L}_{\mathcal{S}}(h)=0
\})
\leq
\sum_{h\in\mathcal{H}_B}
\mathcal{D}^m(\{
x|x\in \mathcal{S},
\mathcal{L}_{\mathcal{S}}(h)=0
\})
$$
其中我们将$\mathcal{S}$拆开,
$$
\mathcal{D}^m(\{
x|x\in \mathcal{S},
\mathcal{L}_{\mathcal{S}}(h)=0
\}
=\mathcal{D}^m(\{
x_i|h(x_i)=f(x_i),
x_i\in \mathcal{S},
i=1,...m
\})
$$
$$
\mathcal{D}^m(\{
x|x\in \mathcal{S},
\mathcal{L}_{\mathcal{S}}(h)=0
\}
=\prod^m_{i=1}\mathcal{D}(\{
x_i|h(x_i)=f(x_i),
x_i\in\mathcal{S}
\})
$$
根据$1-\epsilon\leq e^{-\epsilon}$,
对于等号右边的连乘的每一项都有:
$$
\mathcal{D}(\{
x_i|h(x_i)=y_i,
x_i\in\mathcal{S}
\})
=1-\mathcal{L}_{\mathcal{D},f}(h)
\leq 1-\epsilon
$$
对于所有的$h\in\mathcal{H}_B$:
$$
\mathcal{D}^m(\{
x|x\in \mathcal{S},
\mathcal{L}_{\mathcal{S}}(h)=0
\}
\leq (1-\epsilon)^m
\leq e^{-\epsilon m}
$$
可得:
$$
\mathcal{D}^m
(\{x_i|x_i\in\mathcal{S},i=1,...m,
\mathcal{L}_{\mathcal{D},f}(h_{\mathcal{S}})>\epsilon
\})
\leq |\mathcal{H}_B|e^{-\epsilon m}
\leq |\mathcal{H}|e^{-\epsilon m}
$$
所以,若我们对采样失败概率的容忍大于采样失败的概率上限,
我们认为是最后得到的$h_{\mathcal{S}}$学到了有用的的东西的,
记为:
$$
|\mathcal{H}|e^{-\epsilon m}
\leq \delta
$$
借此我们可以推断出我们所需要拥有的样本数量为:
$$
m \geq \frac{\ln(|\mathcal{H}|/\delta)}{\epsilon}
$$

\textbf{
推论1.1 设$\mathcal{H}$为一个有限假设集合,
$\delta\in(0,1)$,$\epsilon>0$,当
$$
m \geq \frac{\ln(|\mathcal{H}|/\delta)}{\epsilon}
$$
成立时,从而对于任何标记函数$f$、任何分布$\mathcal{D}$,
可实现性假设最少以$1-\delta$的概率,对于每个$ERM$假设$h_{\mathcal{S}}$,
有以下不等式成立:
$$
\mathcal{L}_{\mathcal{D},f}(h_{\mathcal{S}})\leq\epsilon
$$
}

上述推论表明,对于足够大的$m$,由${ERM}_{\mathcal{H}}$规则
生成的有限假设会以$1-\delta$概率得到小于误差$\epsilon$的近似正确解,
概率在于容忍采样失败概率$\delta$,
近似在于容忍误差$\epsilon$,
即我们的算法,
可能(容忍了失败概率)会学到令我们基本满意(容忍了一定误差)的东西。
而概率近似正确,即PAC理论,将在第二章详述。

\subsection{思路总结}
\begin{enumerate}
\item 为什么不能有效学习?为什么会$\mathcal{L}_{\mathcal{D},f}(h)>\epsilon$?
\item 因为样本不具有代表性,学到的东西有问题,采样失败。
\item 怎么解决?
\item 降低采样失败概率至我们可以容忍的范围,小于$\delta$。
\item 采样失败的概率上限小于$\delta$。
\item $\sup{\mathcal{D}^m
(\{x_i|x_i\in\mathcal{S},i=1,...m,
\mathcal{L}_{\mathcal{D},f}(h_{\mathcal{S}})>\epsilon
\})}=|\mathcal{H}|e^{-\epsilon m}\leq \delta$
\item $m \geq \frac{\ln(|\mathcal{H}|/\delta)}{\epsilon}$
\item 当样本足够充足时,学习到的假设概率近似正确。
\end{enumerate}

\newpage


% ============================================================
%
%			第二章 Probably Approximately Correct
%
% ============================================================
\section{Probably Approximately Correct}
\subsection{PAC学习理论}
\subsubsection{PAC可学习}
接着上一章继续说,
在经验风险最小化准则下,
对于一个有限假设类,
如果有足够多的训练样本,
则我们输出的学习算法在真实世界中,
是概率近似正确的。我们做以下定义:

\textbf{
定义2.1
\textit{PAC可学习}
若存在一个函数$m_{\mathcal{H}}:{(0,1)}^2\rightarrow\mathcal{N}$
和一个学习算法,使得对于给定的$\epsilon,\delta\in(0,1)$
和任一分布$\mathcal{D}$、
任一标记函数$f:\mathcal{X}\rightarrow\{0,1\}$,
可实现假设成立时,
那么当样本数量$m\geq m_{\mathcal{H}}(\epsilon,\delta)$时,
算法将以不小于$1-\delta$的概率返回一个使
$$\mathcal{L}_{\mathcal{D},f}(h)\leq\epsilon$$
的假设$h$。
}

\subsubsection{采样复杂度}
函数$m_{\mathcal{H}}:(0,1)^2\rightarrow\mathcal{N}$
决定了假设的采样复杂度,即可以学到东西时所需要的样本数量。
采样复杂度不仅依赖于$\epsilon$和$\delta$,
同样还依赖于假设空间中的假设数量:
$$
m_{\mathcal{H}}=\frac{\ln(|\mathcal{H}|/\delta)}{\epsilon}
$$
其与假设数量的对数成正比。

\textbf{
推论2.1
当采样复杂度满足:
$$
m_{\mathcal{H}}(\epsilon,\delta)
\leq\lceil 
\frac{\ln(|\mathcal{H}|/\delta)}{\epsilon}
\rceil
$$
则任一有限类假设是PAC可学习的。
}

$m_{\mathcal{H}}(\epsilon,\delta)$被称为
假设结合$\mathcal{H}$的采样复杂度的最小函数。
由上述定义我们可知,一个假设是否是PAC可学习的,
还依赖于假设空间的大小,
而假设空间则与VC维相关,将会在后几章说明。

\subsection{泛化PAC理论}
为了让该理论更贴近实际,
则我们考虑将理论的前提约束放宽,进行泛化。
\begin{enumerate}
\item 去掉可实现性假设。
\item 考虑多分类、回归等问题。
\end{enumerate}

\subsection{不可知PAC学习}
当我们放弃可实现假设时,我们称之为不可知,
具体可以理解为:
完全一样的样本,却又不同的标记,这种情况下,
我们的算法如何学到东西。

从此处开始,为了简写,我们将
$\mathcal{D}$定义为$\mathcal{X}\times\mathcal{Y}$上的概率分布,
即将之前$\mathcal{D}$和$f$简写在一起,
作为领域集和标签集的联合概率分布,定义真实误差为:
$$
\mathcal{L}_{\mathcal{D}}(h)
\xlongequal{def}
\mathbb{P}_{(x,y)\sim\mathcal{D}}[h(x)\neq y]
\xlongequal{def}
\mathcal{D}(\{(x,y):h(x)\neq y\})
$$

在面对不可知的情况下,最好的预测器为贝叶斯预测器:
$$
f_{\mathcal{D}}(x)=
\begin{cases}
1& \mathbb{P}[y=1|x]\geq\frac{1}{2}\\
0& else
\end{cases}
$$
我们易证贝叶斯预测器是最优的:
对于每个领域集的实例$x\in\mathcal{X}$
都选择数量较多的类别,
即每个实例的分类正确概率都大于$1/2$,
没有其他分类器可以达到比它更低的错误率,对于任意预测器$g$都有
$\mathcal{L}_{\mathcal{D}}(f_{\mathcal{D}})
\leq \mathcal{L}_{\mathcal{D}}(g)$,
所以该预测器最优。
但我们并不知道实际的概率分布$\mathcal{D}$,
所以我们并没法使用这样的预测器。

所以对于不可知问题,我们只能选择容忍
(看来统计的机器学习就是一个容忍的过程Orz):

\textbf{
定义2.2 \textit{不可知PAC可学习}
若存在一个函数$m_{\mathcal{H}}:{(0,1)}^2\rightarrow\mathcal{N}$和
一个学习算法,
使得对于给定的$\epsilon,\delta\in(0,1)$
和任一分布$\mathcal{D}$,
当样本数量$m\geq m_{\mathcal{H}}(\epsilon,\delta)$时,
算法将以不小于$1-\delta$的概率返回一个使
$$\mathcal{L}_{\mathcal{D}}(h)\leq
\min_{h'\in\mathcal{H}}\mathcal{L}_{\mathcal{D}}(h')+\epsilon
$$
成立的假设$h$。
}

\subsection{学习问题建模}
此节我们讨论学习问题建模,总的来说,除了二分类问题,
学习任务还分为以下几种:
\begin{itemize}
\item 多分类
\item 回归
\end{itemize}
虽然说学习任务分为多种,但主要区别只在于损失函数不同。

\subsubsection{广义损失函数}
给定任意集合$\mathcal{H}$和定义域
$\mathcal{Z}=\mathcal{X}\times\mathcal{Y}$,令
$\ell$为$\mathcal{H}\times\mathcal{Z}$到非负实数的映射,
记为$\ell:\mathcal{H}\times\mathcal{Z}\rightarrow\mathbb{R}_{+}$,
$\ell$就是损失函数。
现在我们重新定义真实误差和经验误差:
$$
\mathcal{L}_{\mathcal{D}}(h)
\xlongequal{def}
\mathbb{E}_{z\sim\mathcal{D}}[\ell(h,z)]
$$
$$
\mathcal{L}_{\mathcal{S}}(h)
\xlongequal{def}
\frac{1}{m}\sum^m_{i=1}\ell(h,z_i)
$$
同时我们定义常用的损失函数:
\begin{itemize}
\item
0-1损失
$$
\ell_{0-1}(h,(x,y))
\xlongequal{def}
\begin{cases}
0& h(x)=y\\
1& h(x)\neq y
\end{cases}
$$
\item 平方损失
$$
\ell_{sq}(h,(x,y))
\xlongequal{def}
{(h(x)-y)}^2
$$
\end{itemize}

\textbf{
定义2.3 
\textit{广义损失函数下的不可知PAC可学习}
对于集合$\mathcal{Z}$和损失函数
$\ell:\mathcal{H}\times\mathcal{Z}\rightarrow\mathbb{R}_{+}$,
若存在一个函数$m_{\mathcal{H}}:{(0,1)}^2\rightarrow\mathcal{N}$和
一个学习算法,
使得对于给定的$\epsilon,\delta\in(0,1)$
和任一分布$\mathcal{D}$,
当样本数量$m\geq m_{\mathcal{H}}(\epsilon,\delta)$时,
算法将以不小于$1-\delta$的概率返回一个使
$$
\mathcal{L}_{\mathcal{D}}(h)\leq
\min_{h'\in\mathcal{H}}\mathcal{L}_{\mathcal{D}}(h')+\epsilon
$$
成立的假设$h$,其中
$\mathcal{L}_{\mathcal{D}}(h)=\mathbb{E}_{z\sim\mathcal{D}}[\ell(h,z)]$。
}

\subsection{思路总结}
本章主要在于泛化PAC学习的试用范围,
通过更改假设来让这个理论更具有普适性,
更贴近生活现实。
(通俗来说就是通过不断的容忍,
放缩范围,
看来搞统计的都是受啊XD)


\section{附录1:不等式证明}
\subsection{Markov Inequality}
马尔科夫不等式:
$$
\mathbb{P}(X\geq a)\leq \frac{\mathbb{E}(Z)}{a}
$$
其中$X$是非负的随机变量。

证:
$$
\begin{aligned}
\mathbb{E}(X) &=  \int_0^{+\infty}{xf(x)dx} \\
	&= \int_0^{a}{xf(x)dx} + \int_a^{+\infty}{xf(x)dx} \\
	&\leq 
	\int_0^{a}{0f(x)dx} + \int_a^{+\infty}{af(x)dx}\\
	&\leq 0 + \int_a^{+\infty}{af(x)dx} 
   = a\int_a^{+\infty}{f(x)dx} = a\mathbb{P}(X>a)
\end{aligned}
$$
移动一下$a$的位置,不等式得证,
其中第二行到第三行是将积分中的$x$换成积分下限。




\end{document}

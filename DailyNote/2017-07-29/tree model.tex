\documentclass[UTF8]{ctexart}
\CTEXsetup[format={\Large\bfseries}]{section} % 标题靠左
% use package here
\usepackage{mathrsfs}
\usepackage{extarrows}
\usepackage{amsmath}
\usepackage{amssymb}
\DeclareMathOperator*{\argmax}{arg\,max}
\usepackage{enumerate}			
\usepackage{graphicx}				




\title{Ensemble Tree Model With Deep Architecture}
\author{Yuyang Zhang}
\date{2017-07-29} % Activate to display a given date or no date (if empty),
         % otherwise the current date is printed 

\begin{document}
\maketitle
% =========================================================================
%
%					1. Related Work
%
% =========================================================================
\section{Related Work}
\subsection{Introduction of Tree Model}
\begin{itemize}
\item 周志华,2016,机器学习,北京,清华大学出版社,425pp
\item Zhou Z H. Ensemble methods: foundations and algorithms[M]. CRC press, 2012.
\item Zhou Z H. Ensemble learning[J]. Encyclopedia of biometrics, 2015: 411-416.
\end{itemize}

\subsection{Paper List}
\begin{itemize}
\item Kontschieder P, Fiterau M, Criminisi A, et al. Deep neural decision forests[C]//Proceedings of the IEEE International Conference on Computer Vision. 2015: 1467-1475.
\item Zhou Z H, Feng J. Deep forest: Towards an alternative to deep neural networks[J]. arXiv preprint arXiv:1702.08835, 2017.
\end{itemize}

% =========================================================================
%
%					1. Related Work
%
% =========================================================================
\section{Deep Neural Decision Forests}
这篇论文提出一种基于概率的树模型,
同时这种模型可以利用BP进行训练,
这应该是它的一大特点。

定义输入为$\mathcal{X}$和输出$\mathcal{Y}$,
决策树的内部决策节点为$n\in\mathcal{N}$,
叶子节点为预测节点,记为$\ell\in\mathcal{L}$,
对于每一个内部节点都有一个决策函数
$$
d_n(x;\theta):\mathcal{X}\rightarrow[0,1]
$$
在标准决策树中,每个决策节点都是二分的,
而且决策标准是确定的,
但本篇论文提出的模型中,
决策节点的输出是一个二项分布的随机变量,
该随机变量的均值为$d_n(x;\theta)$,
因此叶节点的预测值为到达该叶节点的样本概率值,
我们记为
$$
\mathbb{P}_T[y|x,\theta,\pi]=
\sum_{\ell\in\mathcal{L}}\pi_{\ell y}\mu_{\ell}(x|\theta)
$$
其中$\pi=(\pi_{\ell})_{\ell\in\mathcal{L}}$,
$\pi_{\ell y}$表示拥有$y$标签的样本到达叶子节点$\ell$的概率,
$\mu_{\ell}(x|\theta)$是选路函数(routing function),
表明样本$x$到达叶子节点$\ell$的概率,
其满足
$$
\sum_{\ell}\mu_{\ell}(x|\theta)=1
$$
我们对选路函数作详细的定义,
$\ell\swarrow n$和
$n\searrow\ell$分别表示
在决策节点表示叶子节点$\ell$属于节点$n$的左子树和右子树,
所以选路函数可以写作
$$
\mu_{\ell}(x|\theta)=
\prod_{n\in\mathcal{N}}
d_n(x;\theta)^{\mathbb{I}_{\ell\swarrow n}}
\bar{d}_n(x;\theta)^{\mathbb{I}_{n\searrow\ell}}
$$
其中$\bar{d}_n(x;\theta)=1-d_n(x;\theta)$,
$\mathbb{I}_P$是关于条件$P$的指示函数。

\begin{figure}[htbp]
	\small
	\centering
	\includegraphics[width=7cm]{1.png}
	\caption{树结构示意图}
		\label{1}
\end{figure}

如图1所示,$\mu_{\ell_4}=d_1(x)\bar{d}_2(x)\bar{d}_5(x)$。
此处注意,$\mathbb{I}_{\ell\swarrow n}$和$\mathbb{I}_{n\searrow\ell}$
可能同时为$0$。

对于决策节点,定义
$$
d_n(x;\theta)=\sigma(f_n(x;\theta))
$$
其中$\sigma(x)=(1+e^{-x})^{-1}$,
并且要求$f_n(x;\theta):\mathcal{X}\rightarrow\mathbb{R}$是一个实数的映射,
可以认为这是神经网络单元的一种,把$f_n$定义为为一个线性单元,
也可以做一些别的定义。
这样的一棵树也可以做集成学习,
定义森林为$\mathcal{F}={T_1,...,T_k}$是一个森林学习器,
则它的预测输出为
$$
\mathbb{P}_{\mathcal{F}}[y|x]=
\frac{1}{k}\sum^k_{h=1}\mathbb{P}_{T_h}[y|x]
$$

这样一个树模型,可以通过BP的方式进行学习,
把每个树的输出看作概率输出,
定义数据集为$\mathcal{T}\subset\mathcal{X}\times\mathcal{Y}$,
求解其极大似然估计,
即优化对数损失,
则误差为
$$
\mathcal{R}(\theta,\pi;\mathcal{T})=
\frac{1}{|\mathcal{T}|}
\sum_{(x,y)\in\mathcal{T}}
L(\theta,\pi;x,y)
$$
其中损失函数为
$$
L(\theta,\pi;x,y) = -\log(\mathbb{P}_{T}[y|x,\theta,\pi])
$$

按照论文中的说法:
All decision functions depend on a common parameter $\theta$,
which in turn parametrizes each function $f_n$.
所有的$f_n$使用相同的参数,
我感觉此处不是十分合理,
但别人论文里就是这么写的,
可能效果好吧,或者为了减少参数计算量。
优化该目标函数同样采用和神经网络一样的SGD,
我们分为两部分:
优化$\theta$和$\pi$。
对于$\theta$有
$$
\begin{aligned}
\theta^{(t+1)}&=
\theta^{(t)}-\eta
\frac{\partial}{\partial\theta}\mathcal{R}(\theta^{(t)},\pi;\mathcal{B})\\
&=\theta^{(t)}-
\frac{\eta}{|\mathcal{B}|}\sum_{(x,y)\in\mathcal{B}}
\frac{\partial}{\partial\theta}L(\theta^{(t)},\pi;x,y)\\
\end{aligned}
$$
其中$\eta>0$为学习率,$\mathcal{B}$为一个数据随机子集(a mini-batch)。
损失$L$的梯度可以表示为
$$
\frac{\partial}{\partial\theta}L(\theta,\pi;x,y)=
\sum_{n\in\mathcal{N}}
\frac{\partial L(\theta,\pi;x,y)}{\partial f_n(x;\theta)}
\frac{\partial f_n(x;\theta)}{\partial \theta}
$$
我们重点关注一下前面那项:
$$
\begin{aligned}
\frac{\partial L(\theta,\pi;x,y)}{\partial f_n(x;\theta)}
&= \frac{\partial}{\partial f_n(x;\theta)}
\{-\log(\mathbb{P}_{T}[y|x,\theta,\pi])\}\\
&= -\frac{1}{\mathbb{P}_{T}[y|x,\theta,\pi])}
\frac{\partial \mathbb{P}_{T}[y|x,\theta,\pi])}{\partial f_n(x;\theta)}\\
&= -\frac{1}{\mathbb{P}_{T}[y|x,\theta,\pi])}
\frac{\partial\sum_{\ell\in\mathcal{L}}\pi_{\ell y}\mu_{\ell}(x|\theta)}{\partial f_n(x;\theta)}
\\
\end{aligned}
$$
定义$\mathcal{L}_m\subset\mathcal{L}$表示以节点$m$为根节点的子树,则有
$$
\begin{aligned}
\frac{\partial L(\theta,\pi;x,y)}{\partial f_n(x;\theta)}
&= -\frac{\sum_{\ell\in\mathcal{L}_n}\pi_{\ell y}}{\mathbb{P}_{T}[y|x,\theta,\pi])}
\frac{\partial\mu_{\ell}(x|\theta)}{\partial f_n(x;\theta)}\\
&=-\frac{\sum_{\ell\in\mathcal{L}_n}\pi_{\ell y}}{\mathbb{P}_{T}[y|x,\theta,\pi])}
\frac{
\partial
\prod_{m\in\mathcal{N}}
d_m(x;\theta)^{\mathbb{I}_{\ell\swarrow m}}
\bar{d}_m(x;\theta)^{\mathbb{I}_{m\searrow\ell}}
}
{\partial f_n(x;\theta)}\\
&=\frac{
\sum_{\ell\in\mathcal{L}_n}\pi_{\ell y}
\prod_{m\in\mathcal{\bar{N}}_n}
d_m(x;\theta)^{\mathbb{I}_{\ell\swarrow m}}
\bar{d}_m(x;\theta)^{\mathbb{I}_{m\searrow\ell}}
}{\mathbb{P}_{T}[y|x,\theta,\pi])}
\frac{
\partial
d_n(x;\theta)^{\mathbb{I}_{\ell\swarrow n}}
\bar{d}_n(x;\theta)^{\mathbb{I}_{n\searrow\ell}}
}
{\partial f_n(x;\theta)}\\
\end{aligned}
$$
我们令$f_n(x;\theta)=z_n$,同时只关心最后一项时,有
$$
\begin{aligned}
\frac{
\partial
d_n(x;\theta)^{\mathbb{I}_{\ell\swarrow n}}
\bar{d}_n(x;\theta)^{\mathbb{I}_{n\searrow\ell}}
}
{\partial f_n(x;\theta)}
&=\frac{\partial}{\partial z_n}
d_n(z_n)^{\mathbb{I}_{\ell\swarrow n}}
\bar{d}_n(z_n)^{\mathbb{I}_{n\searrow\ell}}
\\
&=\frac{\partial}{\partial z_n}
[\sigma(z_n)^{\mathbb{I}_{\ell\swarrow n}}
(1-\sigma(z_n))^{\mathbb{I}_{n\searrow\ell}}]
\\
\end{aligned}
$$
其中当$\mathbb{I}_{\ell\swarrow n}=1$时
$$
\begin{aligned}
\frac{\partial}{\partial z_n}
[\sigma(z_n)^{\mathbb{I}_{\ell\swarrow n}}
(1-\sigma(z_n))^{\mathbb{I}_{n\searrow\ell}}]
&=
\sigma(z_n)(1-\sigma(z_n))
\\
\end{aligned}
$$
当$\mathbb{I}_{n\searrow\ell}=1$时
$$
\begin{aligned}
\frac{\partial}{\partial z_n}
[\sigma(z_n)^{\mathbb{I}_{\ell\swarrow n}}
(1-\sigma(z_n))^{\mathbb{I}_{n\searrow\ell}}]
&=
-\sigma(z_n)(1-\sigma(z_n))\\
\end{aligned}
$$
我们可以发现,原式基本没变,所以将其带回,我们有
$$
\frac{\partial L(\theta,\pi;x,y)}{\partial f_n(x;\theta)}
=
\frac{
\sum_{\ell\in\mathcal{L}_n}\pi_{\ell y}
\mu_{\ell}(x|\theta)
}{\mathbb{P}_{T}[y|x,\theta,\pi])}
d_n(x;\theta)^{\mathbb{I}_{n\searrow\ell}}
(-\bar{d}_n(x;\theta))^{\mathbb{I}_{\ell\swarrow n}}
$$
在推导验证此处公式的时候,不要忘记$\log$前面的负号,
同时真的也让我理解了Sigmoid函数有多方便...

对于预测的叶子节点概率分布$\pi$的优化也很简单,该问题可以写作
$$
min_{\pi}\mathcal{R}(\theta,\pi;\mathcal{T})
$$
迭代更新公式为
$$
\pi^{(t+1)}_{\ell y}=
\frac{1}{Z^{(t)}_\ell}
\sum_{(x,y')\in\mathcal{T}}
\frac{
\mathbb{I}_{y=y'}\pi^{(t)}_{(\ell y)}\mu_{\ell}(x|\theta)
}{\mathbb{P}_{T}[y|x,\theta,\pi^{(t)}])}
$$
其中$Z^{(t)}_\ell$为归一化因子。

同时,我们也可以同时训练多棵树做集成学习,
文中的说法是所有树可以共享$\theta$参数,
同时每棵树也可以选择不同的$f_n$和$\pi$。
初始化时比较随意,均匀分布或者别的随机数都可以,
具体的训练算法如下图所示。

\begin{figure}[htbp]
	\small
	\centering
	\includegraphics[width=7cm]{2.png}
	\caption{概率树的BP训练算法}
		\label{1}
\end{figure}

简单来讲就是初始化参数后先计算分布$\pi$,再通过训练数据迭代更新$\theta$,
重复$n$次。在训练时候,初始化和具体训练有些小技巧,可以去看原论文。
这篇论文中的实验是做的图像分类,与CNN等模型做了对比。

总的来说,我个人感觉这个模型的优点有以下:
\begin{itemize}
\item 
训练参数少,迭代次数少,训练开销小
\item
思路好,传统的tree model都是在节点上确定分类,最后输出,
本文在树的节点上就比较具有随机性,应该泛化能力会比较强
\item
提出一种基于BP训练的tree model,让人眼前一亮,
虽然内部还是用了方便求导的激活函数,还是有点神经网络模型的意思
\end{itemize}
说完好的,再说说我质疑的
\begin{itemize}
\item
模型效果应该不会特别特别好,
相比于其它深度学习模型,较少的参数决定了模型性能的上限
\item
我感觉这个东西训练不会收敛啊,这么做模型感觉没什么理由,
完全像是一个工程上的做法,或者一种突发奇想的尝试,
不过我也没自己实现,只是有点怀疑
\end{itemize}

总的来说,这篇论文发表于2015年,是深度学习比较火热的时候,
也算是对有深度的集成学习模型的尝试吧,
而且做了一种能用BP训练的树模型,这点的确有点牛逼。


\section{Deep Forest: Towards an Alternative to Deep Neural Networks}
这篇文章解读分析的人就比较多了,
各种大神也都说过了,
具体详情转步知乎搜索周志华的gcForest,
我这里就不细说了,
主要是以下两张图:

\begin{figure}[htbp]
	\small
	\centering
	\includegraphics[width=7cm]{3.png}
	\caption{gcForest模型结构}
		\label{1}
\end{figure}

\begin{figure}[htbp]
	\small
	\centering
	\includegraphics[width=7cm]{4.png}
	\caption{gcForest结果生成}
		\label{1}
\end{figure}
图一表明了gcForest模型的训练结构,
层联级的结构,每层都是多个集成学习模型,
随机森林这种,然后每一层的输出作为下一层的输入,
外加最初始的输入一并进入下一层,
然后模型可以通过验证自己决定层数深度,
当生成的模型效果不够好的时候会终止生长,
这也是集成学习的核心思想之一,好而不同的模型集成。

这篇论文是周老师今年的新作之一,应该已经酝酿已久了,
说说我感觉的优点吧:
\begin{itemize}
\item
在“深度学习的时代”做出了一个深度的集成学习模型,
而且某种程度上来讲,是有一定“深度”的
\item
给我感觉是一种boosting和bagging的结合,
既有串行的层联级结构,也有并行训练的并行结构,
结合了集成学习的两个大方向
\item
深度学习时代不忘初心的表现吧,
既能吸取深度学习的精华,
也不忘传统集成学习的模型
\item
给大家提供了另一种思路,
深度不一定都需要BP来训练,
交叉验证也可以作为训练的依据
\item
自增长的结构,通过模型表现控制模型深度,
控制模型的规模,适应不同大小的数据
\end{itemize}
再说说质疑的点:
\begin{itemize}
\item
首先还是模型能力的问题,
我记得该论文的一个作者说过,
训练这样的模型还是比较浪费资源的,
相当于每层都要计算所有的数据,
然后还要把结果都保留下来,
与深度学习的BP相比,还是有一定劣势,
当把足够大的数据喂给模型的时候,
是否能取得足够好的表现,
还是要打个问号的
\item
无法使用BP,给我一种很难有全局优化的感觉,
只是通过不断的训练来学习数据的表达而已,
而且这种表达,并不是通过全局优化学习到的,
有可能很多种表示到最后都可以达到最优,
但是层与层之间学习到的方向不一样,
这样对模型最后的效果应该会打折扣
\item
我没有仔细去看源码,
但我感觉可以考虑多加每层的模型数量,
减少一下深度,
这样可以多一些并行的训练,
效率应该是会高一点
\end{itemize}

\section{Summary}
总结一下,这两篇论文都是在深度学习火热的情况,
提出的具有“深度”的树模型,或者说集成学习模型,
虽然可能在表现上比不过深度学习的某些模型,但是作为研究,
都是属于那种让人眼前一亮的论文,
新的思路总是能给人很多启发。


\end{document}
